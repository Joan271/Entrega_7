%% PAQUETES DISPONIBLES
\usepackage[utf8]{inputenc} %codificación 
\usepackage[spanish, es-lcroman]{babel} %idioma
\usepackage{float}% para usar H
\usepackage{amsmath} %matemáticas
\usepackage{empheq} 
\usepackage{amsthm}
\usepackage{nccmath}
\usepackage{mathrsfs} %% curly letters 
\usepackage{mathtools} %matemáticas
\usepackage{physics} %diferenciales
\usepackage{blindtext}%texto de relleno
\usepackage{amsfonts} %matemáticas
\usepackage{amssymb} %matemáticas
\usepackage{makeidx} 
\usepackage{graphicx} %imágenes
\usepackage[a4paper, margin=2cm, scale=0.8]{geometry}
\usepackage{wrapfig} %figuras alrededor de texto
\usepackage{caption} %subtítulos de figuras
\usepackage{setspace}%interlineado 
\usepackage{array}% entorno array
\usepackage{ragged2e} %alineación 
\usepackage{tabularx} %entorno tabularx (tablas con ancho fijo)
\usepackage{fancyhdr} %cabeceros y pies de página
\usepackage{multirow} %combinar columnas y filas en una tabla
\usepackage{longtable} %mediante este paquete podemos separar una tabla larga en varias que ocupen las páginas necesarias.
\usepackage[table]{xcolor} %con este paquete cambiamos el color de los objetos y concretamente de la opción de las tablas.
\usepackage{enumitem}%enumeraciones personalizadas
\usepackage{subcaption}%subtítulos de figuras y subfiguras
\usepackage{hyperref} % este paquete siempre debemos colocarlo al final 
\usepackage{multicol} %columnas
\usepackage{parskip} %espaciado entre párrafos
\usepackage[rightcaption]{sidecap} %subtítulos alrededor de figuras
\usepackage{ifthen} %condicionales
\usepackage{xparse} % Para manejar parámetros opcionales
\spanishdecimal{.}
% FORMATO PARA FIGURAS Y SUBFIGURAS
\DeclareCaptionFormat{format1}{
%#1=label, #2 = separator, #3 = text
	\textbf{#1#2}#3
}

\DeclareCaptionFormat{format2}{
%#1=label, #2 = separator, #3 = text
\textsc{#1#2}#3
}  

\DeclareCaptionLabelSeparator{parentesis}{$)\,$ }
\DeclareCaptionStyle{subfigura}{
format = format1, 
justification = raggedright, 
labelsep = parentesis,
font = {color = black, huge},
}
\DeclareCaptionStyle{subfigura2}{
format = format1, 
justification = raggedright, 
labelsep = parentesis,
font = {color = black, Large},
}
%% COMANDO PARA TITULOS
\NewDocumentCommand{\settitle}{O{0} O{0} m}{
	% [nombre del autor][nombre de la asignatura]{titulo}
\ifthenelse{\equal{#1}{0}}{}{\author{#1}}
\ifthenelse{\equal{#2}{0}}
	{
	\date{\vspace*{1em}
	\hrule
	\vspace*{1em}}}
	{\date{\Large{UAM - #2}\vspace*{1em}
	\hrule
	\vspace*{1em}}
	}
\title{\Huge \textbf{#3}}
}

%% COMANDO PARA ELEGIR FORMATO
\NewDocumentCommand{\setformat}{O{0.8\textwidth} O{centering} m}
% [ancho de figuras][alineacion de figuras]{cabecero derecho}
% tras elegir este comando no olvidar usar \brightmode y \darkmode para cambiar aplicar el formato deseado
{
%FORMATO PARA PÁGINAS OSCURAS Y CLARAS
\fancypagestyle{claroi}{
\fancyhf[]{}
\fancyhead[L]{\color{black}\textsc{UAM}}
\fancyhead[R] {\color {black}\textsc{#3}}
\fancyfoot[C]{\color{black}\thepage}
\setlength{\headheight}{18pt} 
\renewcommand{\headrulewidth}{0.75pt}
\renewcommand{\headruleskip}{-0.5em}
\renewcommand{\headrule}{\hbox to\headwidth{%
    \color{black}\leaders\hrule height \headrulewidth\hfill}}
  \renewcommand{\footrulewidth}{0pt}
}

\fancypagestyle{oscuroi}{
\fancyhf[]{}
\fancyhead[L]{\color{white}\textsc{UAM}}
\fancyhead[R] {\color {white}\textsc{#3}}
\fancyfoot[C]{\color{white}\thepage}
\renewcommand{\headrulewidth}{0.75pt}
\renewcommand{\headruleskip}{-0.5em}
\renewcommand{\headrule}{\hbox to\headwidth{%
    \color{white}\leaders\hrule height \headrulewidth\hfill}}
  \renewcommand{\footrulewidth}{0pt}
}
\DeclareCaptionStyle{figura_claro_custom}{
format = format1, 
justification = #2,
font = {color = black,small},
name = Fig.,
width = #1
}

\DeclareCaptionStyle{figura_oscuro_custom}{
format = format1, 
justification = #2,
font = {color = white,small},
name = Fig.,
width = #1
}

\DeclareCaptionStyle{tabla_claro_custom}{
format = format1, 
justification = #2,
font = {color = black,small},
name = Tabla,
width = #1
}

\DeclareCaptionStyle{tabla_oscuro_custom}{
format = format1, 
justification = #2,
font = {color = white,small},
name = Tabla,
width = #1
}


% MODO OSCURO O CLARO
\def\darkmode{
\pagecolor{black}
\color{white}
\pagestyle{oscuroi}
\captionsetup[figure]{style=figura_oscuro_custom}
\captionsetup[table]{style = tabla_oscuro_custom}
\hypersetup{
colorlinks = true,%% atento a la separación con comas
%% si colocas colorlinks = false aparecen cajas alrededor de los links pero no se ven ni con texmaker ni los navegadores convencionales,en cambio sí que se ve con adobe acrobat reader. 
linkcolor = cyan, 
citecolor= cyan,
filecolor = magenta, 
urlcolor = cyan,
pdfpagemode = FullScreen,
urlbordercolor = {1 0 0},
linktocpage = false, %% si es verdadero son las páginas del índice las que quedan referenciadas
}
}
\def\brightmode{
\pagecolor{white}
\color{black}
\pagestyle{claroi}
\captionsetup[figure]{style=figura_claro_custom}
\captionsetup[table]{style = tabla_claro_custom}
}
\hypersetup{
colorlinks = true,%% atento a la separación con comas
%% si colocas colorlinks = false aparecen cajas alrededor de los links pero no se ven ni con texmaker ni los navegadores convencionales,en cambio sí que se ve con adobe acrobat reader. 
linkcolor = blue, 
citecolor= blue,
filecolor = magenta, 
urlcolor = blue,
pdfpagemode = FullScreen,
urlbordercolor = {1 0 0},
linktocpage = false, %% si es verdadero son las páginas del índice las que quedan referenciadas
}
}
\urlstyle{same}
\definecolor{coolgreen}{RGB}{15, 219, 97}
%QUITAR NÚMERO EN UNA PÁGINA
\def\resetnumpagetitle{
\thispagestyle{empty}
\newpage
\setcounter{page}{1}
}
%% COMANDOS MATEMÁTICOS
\DeclareMathOperator{\rot}{\textrm{\textbf{rot}}}
\DeclareMathOperator{\diver}{\textrm{\textbf{div}}}
\renewcommand{\grad}{\textrm{\textbf{grad}}} 
\newcommand{\appeq}{\backsimeq} %%aproximaciones
\newcommand{\FT}[1]{\mathcal{F}\{#1\}} % Transformada de Fourier 
\newcommand{\LT}[1]{\mathcal{L}\left\{ #1 \right\}} %Transformada de Laplace
\newcommand{\BLT}[1]{\mathcal{B}\left\{#1\right\}} % Transformada bilateral de Laplaca
\newcommand{\mean}[1]{\left \langle#1\right \rangle} % medias
\newcommand{\Oterm}[2]{\mathcal{O}\qty(#1)^#2} % big O notation

%% COMANDO PARA RESOLUCIÓN
\newcommand{\Resolucion}{\vspace*{1em}
\hrule
\vspace*{1em}
{\color{blue} Resolución:}\\

}


